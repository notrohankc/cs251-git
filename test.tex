\documentclass{article}

\begin{document}
\title{MovieReview of Pyaasa}


\centerline{\sc \large CS251}
\vspace{.5pc}
\centerline{\sc }
\vspace{2pc}
\title{Payaasa 1957}

Pyaasa does show its age. There are a lot of things that would perplex or annoy the modern viewer such as jumps in continuity and a simplistic set of characters. However, if one looks past that, then the movie reveals a rich story and a deep message. This is even more astonishing considering that the movie was made in the 50s.

This film is a journey into the mind of Guru Dutt and shows the hurt that sensitive minds, such as those of poets', can feel in modern society. As such, its message transcends time and is still relevant today. Those with less sensitive dispositions, such as mine, will not feel left out and will surely appreciate the story too.

The dialogue is beautiful and the songs are sublime. This is a real pity for those who do not understand Hindi since the subtitles (although decent) rob the words of much of their beauty. I know from other reviews on this site that non-Indian viewers often rightly skip the much-disliked song sequences. I still recommend that they sit through them with the subtitles, of course, turned on. This will convey some sense of the beauty of the songs.

There are several moments when the movie seems idiosyncratic. On both occasions that Abdul Sattar (Johnny Walker) broke out into song, the sudden "change" in his voice seemed absurd. Similarly, several bits of dialogue were almost comical when they were meant to be serious, especially in the case of the first publisher. The video itself looks jaded and the sound leaves a lot to be desired. But as I stated earlier, one must look past this to truly enjoy this splendid movie.

One of the striking things about the movie that I must mention is how much India seems to have changed since 1957. Perhaps this is because of the way the movie was shot. The Indian city of the 50s seems very wide and open and even though it is shown teeming with people, there is certain spaciousness about the whole thing. This stands in stark contrast to the clustered urban environment of today.

In all, I would highly recommend this movie to film lovers, both Indian and otherwise.

by rohankc

This is an excellent movie which raises many important issues which include unemployment, prostitution in all its worst colours, the exploitation of writers by publishers, breaking family cords etc. Handling of these problems leaves an indelible effect on the viewer and the film tries to shake the souls of the viewers. The lyrics by Sahir Ludhianwi fit the situations and enhance the effect of the film as usual. Mohammad Rafi also sang the songs in his inimitable style of playback acting which is also adds a gem to the film. The level of acting is also very high and special mention has to be made of Waheeda Rahman, Guru Dutt and Rehman. The handling of important scenes keeps viewers glued to the screen till the end. Actually this is one of the best movies made in India. 

\end{document}
