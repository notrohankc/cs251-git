\documentclass{article}
\usepackage[utf8]{inputenc}

\title{CS251: Assignment1}
\author{Rohit Kr. Bose, Sunil Kumar Pandey, Rohan Krishnan Chaudhary }
\date{April 2017}

\begin{document}

\maketitle

\section{Movie Review: 3 Idiots}
"Whatever the problem in life is... just say to yourself 'Aal Izz Well'.. This wont solve your problems but it will give the courage to face it.." "Chase Excellence and success will follow".. " Life is not about getting marks, grades but chasing your dreams".. These are the golden rules which 3 IDIOTS teaches you in a very light and entertaining way.. The movie makes you laugh and in the process you learn many golden rules which can alter your life in a big manner...

3 actors from the path-breaking blockbuster RANG DE BASANTI( Aamir, Sharman, Madhvan) team up together with Rajkumar Hirani.. It couldn't get bigger! Loosely based on blockbuster novel ' Five Point Someone ' by Chetan Bhagat the movie deals with the present education system in India that whether getting more marks and better grades is better than gaining knowledge and is mugging up everything more useful than understanding it..

The movie has many hilarious scenes and everyone in the film form Aamir to Boman Irani to Omi Vaidya were superb in their roles.. Aamir has outdone himself.. Never in the film he has looked 44 year old... Film's music sounds mediocre but when viewed on the screen then the music sounds perfect.. Cinematography is awesome and so is the direction.. Raju Hirani's screenplay(you can easily ignore a few loose points) is a masterpiece and the film will achieve cult status for engineering students in a short period of time.. It has the soul which Five Point Someone had.. It is easily Aamir Khan's finest film to date.. The film's climax is fantabulous and is a shocker.. It takes everyone by surprise...

To conclude 3 IDIOTS has the potential to break all records and sweep all awards.. Watch it... You will Love it...

My Rating: 8.5/10
\section{Movie Review: Anand}
It is very rare for guys to shed tears after watching a movie, this movie does make you shed tears for Anand, the main protagonist played by the superstar of the 70's Rajesh Khanna.

The movie has everything going for it. Acting, direction, story, music, dialogues etc... everything is fabulous. It has drama, humor, emotions in dollops. It is a story a dying man who looks at life with a positive attitude and enjoys his time knowing fully well his disease is incurable and that he is going to die soon.

Rajesh Khanna as Anand is absolutely brilliant, this is his career-best performance , notwithstanding movies like Kati Patang, Roti, Aradhana, Amar Prem etc. You cannot think of any other actor in this role and to think Rajesh Khanna was not the original choice(Shashi Kapoor was). He makes you laugh and cry. He causes anxiety and goose bumps. Simply superb.

Amitabh Bachchan is fantastic as babumoshai( a name with which Raj Kapoor used to address the film's director Hrishikesh Mukherjee). He showed the world that the next superstar was coming, though he really "arrived" a couple of years later. The rest of the supporting cast is also brilliant be it Johhny Walker(stands out) or Ramesh Deo or Seema or Sumitra. Everyone is wonderful.

Music is the hallmark of all great hindi movies and this one has music ranking right up there, on the top. Be it "Kahin door jab din dhal jaye" or " Maine tere liye hi saath rang" or "Zindagi kaisi hai paheli".

Maverick composer Salil Chowdury comes up with an absolutely fantastic score and singers Mukhesh and Manna Dey do complete justice to his tunes.

About the director Hrishikesh Mukherjee, what can one say, he is one of the best directors ever in the Indian film history. A guy with a complete repertoire,a complete entertainer (though people consider other directors to be more entertaining, but real movie buffs will agree with me). All his movies, he has been directing movies since 1957 are worth a watch. Some are brilliant and others watchable. None of his movies can be rated as unwatchable(except maybe Jooth bole Kauwa kaate and Jhooti).

Simply put, this is one of the best Hindi movies ever made.
\end{document}
