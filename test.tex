\documentclass{article}
\usepackage[utf8]{inputenc}

\title{CS251: Assignment1}
\author{Rohit Kr. Bose, Sunil Kumar Pandey, Rohan Krishnan Chaudhary }
\date{April 2017}

\begin{document}

\maketitle

\section{Movie Review: 3 Idiots}
"Whatever the problem in life is... just say to yourself 'Aal Izz Well'.. This wont solve your problems but it will give the courage to face it.." "Chase Excellence and success will follow".. " Life is not about getting marks, grades but chasing your dreams".. These are the golden rules which 3 IDIOTS teaches you in a very light and entertaining way.. The movie makes you laugh and in the process you learn many golden rules which can alter your life in a big manner...

3 actors from the path-breaking blockbuster RANG DE BASANTI( Aamir, Sharman, Madhvan) team up together with Rajkumar Hirani.. It couldn't get bigger! Loosely based on blockbuster novel ' Five Point Someone ' by Chetan Bhagat the movie deals with the present education system in India that whether getting more marks and better grades is better than gaining knowledge and is mugging up everything more useful than understanding it..

The movie has many hilarious scenes and everyone in the film form Aamir to Boman Irani to Omi Vaidya were superb in their roles.. Aamir has outdone himself.. Never in the film he has looked 44 year old... Film's music sounds mediocre but when viewed on the screen then the music sounds perfect.. Cinematography is awesome and so is the direction.. Raju Hirani's screenplay(you can easily ignore a few loose points) is a masterpiece and the film will achieve cult status for engineering students in a short period of time.. It has the soul which Five Point Someone had.. It is easily Aamir Khan's finest film to date.. The film's climax is fantabulous and is a shocker.. It takes everyone by surprise...

To conclude 3 IDIOTS has the potential to break all records and sweep all awards.. Watch it... You will Love it...

My Rating: 8.5/10
\section{Movie Review: Pyaasa}

Pyaasa does show its age. There are a lot of things that would perplex or annoy the modern viewer such as jumps in continuity and a simplistic set of characters. However, if one looks past that, then the movie reveals a rich story and a deep message. This is even more astonishing considering that the movie was made in the 50s.

This film is a journey into the mind of Guru Dutt and shows the hurt that sensitive minds, such as those of poets', can feel in modern society. As such, its message transcends time and is still relevant today. Those with less sensitive dispositions, such as mine, will not feel left out and will surely appreciate the story too.

The dialogue is beautiful and the songs are sublime. This is a real pity for those who do not understand Hindi since the subtitles (although decent) rob the words of much of their beauty. I know from other reviews on this site that non-Indian viewers often rightly skip the much-disliked song sequences. I still recommend that they sit through them with the subtitles, of course, turned on. This will convey some sense of the beauty of the songs.

There are several moments when the movie seems idiosyncratic. On both occasions that Abdul Sattar (Johnny Walker) broke out into song, the sudden "change" in his voice seemed absurd. Similarly, several bits of dialogue were almost comical when they were meant to be serious, especially in the case of the first publisher. The video itself looks jaded and the sound leaves a lot to be desired. But as I stated earlier, one must look past this to truly enjoy this splendid movie.

One of the striking things about the movie that I must mention is how much India seems to have changed since 1957. Perhaps this is because of the way the movie was shot. The Indian city of the 50s seems very wide and open and even though it is shown teeming with people, there is certain spaciousness about the whole thing. This stands in stark contrast to the clustered urban environment of today.

In all, I would highly recommend this movie to film lovers, both Indian and otherwise.

by rohankc


This is an excellent movie which raises many important issues which include unemployment, prostitution in all its worst colours, the exploitation of writers by publishers, breaking family cords etc. Handling of these problems leaves an indelible effect on the viewer and the film tries to shake the souls of the viewers. The lyrics by Sahir Ludhianwi fit the situations and enhance the effect of the film as usual. Mohammad Rafi also sang the songs in his inimitable style of playback acting which is also adds a gem to the film. The level of acting is also very high and special mention has to be made of Waheeda Rahman, Guru Dutt and Rehman. The handling of important scenes keeps viewers glued to the screen till the end. Actually this is one of the best movies made in India. 


Simply put, this is one of the best Hindi movies ever made.
\end{document}
